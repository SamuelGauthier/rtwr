\section{The Simple Gerstner Wave Approach}\label{sec:simple_gerstner_waves}

% In the simple gerstner wave approach talk on how I build the application
% because SLProject wasn't compiling and say also that 0 A.D. was an overkill to
% start with. Talk about the LSD MDMA strange effects, talk about the parameters
% that have to be shit out of the ass. Talk about the Shading method used. Also
% mention that 16 weeks ago you didn't know anything and that's why you choose
% to implement the gerstner approach.

We opted to implement the Gerstner Wave approach from \cref{subsub:gerstner}
because it was the easiest one to understand and well described in
\autocite{fernando2004gpu}. Furthermore, it does no represent a daunting task
for someone new to the graphics pipeline programming.

We integrated the water model into a demonstration application coded in C++
using OpenGL and the \textit{GLEW}, \textit{GLFW} and \textit{stb} libraries.
C++ was used because both applications in \autoref{subsec:candidate_apps} are
programmed with it. We choose \textit{GLEW} because it is a cross-platform
library to load the OpenGL extensions. \textit{GLFW} is used for the window
creation and keyboard interactions. Finally we utilize the simple but powerful
capabilities of \textit{stb} to read and load images into memory.

We programmed the following classes: \texttt{Camera}, \texttt{Plane},
\texttt{Skybox}, \texttt{FPSCounter}. The \texttt{Camera} class represents the
position of the camera in spherical coordinates\footnote{Our implementation is
    based on the recommendation of the Graphics Stackexchange post:
\url{https://computergraphics.stackexchange.com/questions/151/}}. This makes it
easier to compute the zoom, rotation in the $\varphi$ and $\theta$ directions.
We use the \texttt{Plane} class to build a grid of vertices with the
corresponding indices. The grid always points to the $y$ direction\footnote{In
OpenGL the $z$ coordinate points out of the screen.}, and can have a arbitrary
resolution (amount of vertices) as well as size. This class is used to represent
the water. The \texttt{Skybox} a simply cube whose indices are specified
counterclockwise such that the normals point inwards. The particularity of it,
is that the depth buffer is disabled before the OpenGL draw call is issued and
re-enabled afterwards. Every object rendered after it, will be displayed in
front of it. Finally the \texttt{FPSCounter} class tracks the time needed to
compute each frame.

In the vertex shader of the water plane we compute the position, normal and
tangent vectors as in
\cref{eq:gerstner_position,eq:gerstner_normal,eq:gerstner_tangent} with the only
subtlety that the $y$ and $z$ coordinates are flipped. We use four waves ($i =
4$) with the constants $g = 9.81$, $\kappa = 0.2$, $\pi = 3.14159265358979$ and 
values for $\omega$, $\varphi$, $A$, $Q$ and $\textbf{D}$ as follows:

\begin{equation}\label{eq:gerstner_constants}
\begin{split}
    \omega ={}& \Big\{\sqrt{g 2\pi}, \sqrt{g 4\pi}, \sqrt{g 10\pi}, \sqrt{g
    \pi}\Big\},\\
    %
    \varphi ={}& \{0.4 \omega_1, 0.3 \omega_2, 0.3 \omega_3, 0.1 \omega 4\},\\
    %
    A ={}& \{0.01, 0.008, 0.017, 0.003\},\\
    %
    Q ={}& \Bigg\{\frac{\kappa}{\omega_1 A_1 i}, \frac{\kappa}{\omega_2 A_2 i},
    \frac{\kappa}{\omega_3 A_3 i}, \frac{\kappa}{\omega_4 A_4 i} \Bigg\},\\
    %
    \textbf{D} ={}& \Bigg\{\begin{bmatrix}0.0 \\ 0.5\end{bmatrix},
    \begin{bmatrix}0.8 \\ 0.7\end{bmatrix}, \begin{bmatrix}0.6 \\
    0.6\end{bmatrix}, \begin{bmatrix}0.7 \\ -0.3\end{bmatrix}\Bigg\}
\end{split}
\end{equation}

The implementation can be found in \autoref{lst:vertex} on
page~\pageref{lst:vertex}.

\lstdefinestyle{glsl}{
  belowcaptionskip=1\baselineskip,
  breaklines=true,
  xleftmargin=\parindent,
  language=C,
  showstringspaces=false,
  basicstyle=\footnotesize\ttfamily,
  keywordstyle=\bfseries\color{red!70!black},
  commentstyle=\itshape\color{black!40},
  morekeywords={vec3, vec4, dot, reflect, refract, mix, pow, texture, sin, cos},
  %255 75 62
}
\lstset{style=glsl, caption={Implementation of
\cref{eq:gerstner_position,eq:gerstner_normal,eq:gerstner_tangent}},
label={lst:vertex},captionpos=b}
\begin{figure}[ht!]
\begin{lstlisting}
vec4 position = in_Position;
vec3 normal = vec3(0.0, 1.0, 0.0);
vec3 tangent = vec3(0.0, 0.0, 1.0);

for(int i = 0; i < NUMWAVES; i++) {
    float alpha = w[i] * dot(D[i], position.xz) + phi[i] * t;
    float WA = w[i] * A[i];
    float sinAlpha = sin(alpha);
    float cosAlpha = cos(alpha);
    float DxCos = D[i].x * cosAlpha;
    float DyCos = D[i].y * cosAlpha;
    float DxDy = D[i].x * D[i].y;

    position.x += Qs[i] * A[i] * DxCos;
    position.y += A[i] * sinAlpha;
    position.z += Qs[i] * A[i] * DyCos;

    normal.x -= D[i].x * WA * cosAlpha;
    normal.y -= Qs[i] * WA * sinAlpha;
    normal.z -= D[i].y * WA * cosAlpha;

    tangent.x -= Qs[i] * DxDy * WA * sinAlpha;
    tangent.y += D[i].y * WA * cosAlpha;
    tangent.z -= Qs[i] * D[i].y * D[i].y * WA * sinAlpha;
}
\end{lstlisting}
\end{figure}

We pass to the fragment shader the local position, normal and tangent vectors.
As well as the local light position and the roughness of the surface.

In the fragment shader we update the normal with the values read from the normal
map. Then we transform the light position, vertex position and normal into the
eye space. We compute the reflection and refraction vectors as described in
\autoref{subsec:ocean_details} with the help of glsl's \texttt{reflect(\ldots)}
and \texttt{refract(\ldots)} functions. The two corresponding colors are fetched
from the cube map texture of the skybox and mixed together based on Schlick's
approximation of the Fresnel term. This gives our \texttt{ambient} term, whose
computation is shown in \autoref{lst:fragment_refref}
on~\pageref{lst:fragment_refref}. \texttt{F\_0} results form
\autoref{eq:schlick_cst} with $n_{water} = 1.333$ and equal to 0.020373.
\texttt{RATIO} is the ratio between the water and air refractive indices and
equal to 0.75.

\lstset{style=glsl, caption={Reflection and refraction color computation},
label={lst:fragment_refref},captionpos=b}
\begin{figure}[ht!]
\begin{lstlisting}
vec3 reflected = reflect(incident_eye, n);
reflected = vec3(inverse_V * vec4(reflected, 0.0));

vec3 refracted = refract(incident_eye, n, RATIO);
refracted = vec3(inverse_V * vec4(refracted, 0.0));

vec4 reflectColor = texture(cube_texture, reflected);
vec4 refractColor = texture(cube_texture, refracted);

vec4 ambient = mix(refractColor, reflectColor, F(F0, v, n));
\end{lstlisting}
\end{figure}

For the \texttt{specular} term we use the Cook-Torrance microfacet specular
BRDF\footnote{Bidirectional reflectance distribution function} as
described in \autoref{eq:cook-torrance_spec}.

\begin{equation}\label{eq:cook-torrance_spec}
    f(\textbf{l}, \textbf{v}) ={} \frac{F(\textbf{l}, \textbf{h})
        G(\textbf{l}, \textbf{v}, \textbf{h}) D(\textbf{h})}{4
    (\textbf{n} \cdot \textbf{l}) (\textbf{n} \cdot \textbf{v})}
\end{equation}
%
$F(\textbf{l}, \textbf{h})$ is Schlick's approximation of the Fersnel term as
seen in \autoref{eq:schlick}. For $D(\textbf{h})$, the normal distribution
function, we use the GGX/Trowbridge-Reitz function:

\begin{equation}\label{eq:ndf_ggx}
    D(\textbf{h}) ={} \frac{\alpha^2}{\pi{({(\textbf{n} \cdot \textbf{h})}^2
    (\alpha^2 - 1) + 1)}^2}
\end{equation}
%
$\alpha$ is the squared roughness. And for the specular geometric attenuation
term $G(\textbf{l}, \textbf{v}, \textbf{h})$ we take the height correlated Smith
function:

\begin{equation}\label{eq:height_smith}
    k ={} \frac{\alpha}{2}
\end{equation}
\begin{equation}\label{eq:height_smith}
    G_1(\textbf{x}) ={} \frac{\textbf{n} \cdot \textbf{x}}{(\textbf{n} \cdot
    \textbf{x})(1 - k) + k}
\end{equation}
\begin{equation}\label{eq:height_smith}
    G(\textbf{l}, \textbf{v}, \textbf{h}) ={} G_1(\textbf{l}) G_1(\textbf{v})
\end{equation}

We choose to use the same equations as in \autocite{karis2013real} for
$F(\textbf{l}, \textbf{h})$, $G(\textbf{l}, \textbf{v}, \textbf{h})$ and
$D(\textbf{h})$. The implementation of those functions can be found in
\autoref{lst:fragment_brdf} on page~\pageref{lst:fragment_brdf}. To reduce the
computations we avoid using the \texttt{dot(\ldots)} function inside of them.
Instead we pass the value of the computed dot product.

\lstset{style=glsl, caption={Cook-Torrance microfacet specular BRDF helper
functions}, label={lst:fragment_brdf},captionpos=b}
\begin{figure}[ht!]
\begin{lstlisting}
// Shlick approximation
float F(float F_0, vec3 v, vec3 h) {

    return F_0 + (1 - F_0) * pow((1 - max(dot(v, h), 0)), 5);
}

float G_smith(float nx, float k) {

    return nx / (nx * (1 - k) + k);
}

// Height correlated Smith
float G(float nl, float nv, float k) {

    return G_smith(nl, k) * G_smith(nv, k);
}

// Trowbridge-Reitz
float D(float nh, float alpha2) {

    float denom = nh * nh * (alpha2 - 1) + 1;
    return alpha2 / (PI * denom * denom);
}
\end{lstlisting}
\end{figure}

Finally we output the final fragment color as the sum of the ambient and
specular values: \texttt{out\_Color = specular + ambient}.
