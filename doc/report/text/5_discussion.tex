\section{Discussion}\label{sec:discussion}

% Failed attempt, in future use projected grid as in many productions,
% integrating in 0 a.d. will be difficult because rendering time is unknown,
% targeted devices may not support it? Say that the reflection method used is
% not good, only works for the skybox.
% In the discussion talk about the bad frame rate, the visual unpleasantness and
% the better thing to do, that is projected grid and other method. Also tell
% that's what you will use in the bachelor thesis and give the relevant
% literature with it.

Having seen \autoref{fig:max_raytraced} and other figures from
\autocite{fernando2004gpu} we knew from the start that the result would not be a
realistic ocean water. After implementing the Gerstner water model ourselves, we
now have the confirmation that this model alone is indeed not enough to
represent a realistic water surface. We will discuss two important issues
further below: visual accuracy and performance.

\subsection{Visual Accuracy}\label{subsec:visual_accuracy}

Four waves is in our view insufficient to approximate realistically the shape of
the ocean. We had trouble to find constants which produced acceptable results:
in all the resources explaining the Gerstner model, finding them where ``left to
the artist'' \autocite[Chapter~1]{fernando2004gpu}. We came across one resource
\autocite{wiliams2017tutorial} in which the author implemented the model with
the help of the Unreal Engine but unfortunately her constants did not produce
acceptable results in our application. The rounded shape is only suitable in
cases where it is artistically needed or in those displayed in
\autoref{fig:reference}. 

Our implementation does not take into account the interaction with other objects
on its surface. Hence they are not reflected appropriately. We propose to
compute the reflection with the help of another camera placed below the main one
and looking up to the sky. We then have to make only a texture lookup to get the
reflection fragment color. To render the interactions with the object we could
store them like in \autocite{anno2008meersimulation} into a displacement map and
combine it with another one describing the ocean's shape. Another possibility is
to combine the FFT approach from \cref{subsub:fft} with the \textit{Wave
Particles} from \cref{subsub:wave_particles}. Foam could be blended in based on
the height of the wave, its surrounding slope and on the depth of the water as
described in \cref{subsub:foam}. Caustics could also be faded in
based on the depth of the water. Finally there should be a change in color of
the water based again on its height.

\subsection{Performance}\label{subsec:performance}

58ms to render the plane is much too costly for an application having other
objects moving on the screen. This result is due to the expensive computations
done in the vertex and fragment shader, to the hardware on which we tested our
application and the fact that we did not use any kind of level of detail. The
plane has always the same amount of vertices describing it, no matter if viewed
from far away or up close. Hence it is imperative to implement a level of detail
solution, like the projected grid from \autoref{algo:projected_grid}, or the
radial grid from \autocite[Chapter~18]{pharr2005gpu}.

