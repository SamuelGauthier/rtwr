\section{Water Models}\label{sec:water_models}

% Navier stokes, Different level of water depth, Spatial, Spectral domain,
% eulerian, lagrangian frameworks, talk about their visual accuracy, SLProject,
% 0 a.d.

% Talk about the different depth of water

We start this section by presenting the physically accurate description of
fluids, \nameref{subsec:navier_stokes}, in \autoref{subsec:navier_stokes}. Water
models are separated into groups depending on the water depth: deep,
intermediate or shallow. This classification takes its root from oceanography
\autocite{darles2011survey}. We will present the deep water group in
\autoref{subsec:deep_water} and the shallow one in
\autoref{subsec:shallow_water} (we include the intermediate models directly into
the shallow group). For each of them we will present selected relevant
approaches from the literature. Those two groups describe only the
\textit{shape} of the water. In \autoref{subsec:ocean_details} we discuss the
methods to produce realistic water surfaces. This includes caustics, foam, light
scattering and more. Finally in \autoref{subsec:candidate_apps} we talk about
two applications in need of a real-time water solution. Readers who desire to
get a broader picture about water rendering should read the survey from
\citeauthor{darles2011survey} \autocite{darles2011survey}.


\subsection{Navier-Stokes Equations}\label{subsec:navier_stokes}

% explain that they describe a fluid but they don't have a solution yet and it's
% one of the biggest open problems in math. Computational Fluid Dynamics CFD is
% the field of finding numerical approximations to the equations
% The process of solving problems in fluid dynamics numerically on a computer is
% called CFD

Navier-Stokes Equations (or NSE) describe the motion of a fluid in any dimension
$\mathbb{R}^n$. They are named after Sir George Gabriel Stokes\footnote{Best
known for \textit{Stokes's law}, the expression of the frictional force of a
fluid exerted onto a sphere.}, an Irish mathematician and physicist and Henri
Navier, a French engineer, mathematician and economist. Both worked on, and
enhanced the Euler equations. Navier proposed in 1820 to add a term to them for
the heat dissipation. In 1845 Stokes also suggested introducing a term for the
energy dissipation in form of heat \autocite{gallagher2010autour}. The equations
are shown below only for the curious reader as we will not further discuss them.
The Navier-Stokes equation of incompressible fluid flow is:

\begin{equation}
    \frac{\partial \textbf{u}}{\partial t} + \textbf{u}\cdot \nabla \textbf{u}
     = - \frac{\nabla P }{\rho} + \nu {\nabla}^2 \textbf{u}
\end{equation}

where $\nu$ is the kinematic viscosity, $\textbf{u}$ is the velocity of the
fluid parcel, $P$ is the pressure, $\rho$ is the fluid density and $t$ the
time \autocite{weisstein2018navier}.

Solving the Navier-Stokes equations in three dimensions remains an open problem
in mathematics and physics. It is one of the seven ``Millennium Problems''
defined by the Clay Mathematics Institute.

We felt the need to mention them here because all the water models found in the
literature are somehow linked to them. Computational Fluid Dynamics (CFD) is the
field of finding numerical approximations to those equations. 


\subsection{Deep Water Models}\label{subsec:deep_water}
% What is the spatial domain?  present some papers:
% - Johanson
% What is the spectral domain?  present some models in the spectral and spatial
% domains from different papers that look great:
% - Tessendorf
% - GPU Gems 2 Effective Water Simulation from Physical Models
% - Using Vertex Texture Displacement for Realistic Water Rendering
Deep water models can further be classified into two groups: spatial and
spectral \autocite{darles2011survey}. The former uses a sum of sines and cosines
to describe the water surface whereas the latter uses a inverse fast Fourier
transform (IFFT)\footnote{The FFT approach allows to use data from a buoy or
from aerial water photographs}. Unfortunately both have downsides: spatial
models produce water having a too rounded shape and spectral models are
difficult to control \autocite{darles2011survey}. In order to benefit from both,
a hybrid solution should be chosen.

We present four models, three from the spatial group, one from the spectral and
a hybrid one.

\subsubsection{Projected Grid Concept}
\subsubsection{Gerstner Waves}
\subsubsection{Vertex Texture Displacement}

\subsubsection{Fast Fourier Transform (FFT)}


The Fast Fourier Transform approach was introduced in 1987 by
\citeauthor{mastin1987fourier} \autocite{mastin1987fourier}. The representation
of the wave height $h$ given a horizontal position $\textbf{x} = (x,z)$ and the
time $t$ is given by

\begin{equation}
    h(\textbf{x}, t) = \sum_{k}^{} \tilde{h}(\textbf{k},
    t)e^{i\textbf{k}\cdot\textbf{x}}
\end{equation}

where $\textbf{k}$ is the wave vector and $\tilde{h}(\textbf{k}, t)$ is the
amplitude of the Fourier component obtained from a theoretic wave spectrum
\autocite{tessendorf2001simulating,darles2011survey}.

\citeauthor{tessendorf2001simulating} proposed another amplitude function
$\tilde{h}_0$, 

\begin{equation}\label{eq:tessendorf_h0}
    \tilde{h}_0(\textbf{k}) = \frac{1}{\sqrt{2}}(\xi_r +
    i\xi_i)\sqrt{P_h(\textbf{k})}
\end{equation}

where $\gamma_r$ and $\gamma_i$ are constants and the spectrum $P_h(k)$ is:

\begin{equation}
    P_h(\textbf{k}) = A\frac{e^{\frac{-1}{{(kL)}^2}}}{k^4}
    |\textbf{k}\cdot\textbf{w}|
\end{equation}

\subsubsection{Hybrid}

\subsection{Shallow Water Models}\label{subsec:shallow_water}
% Eulerian and Lagrangian models


\subsection{Rendering of ocean details}\label{subsec:ocean_details}
% Modeling waves and surf 1986
% Nishita et al. [NSTN93] Check the ocean color part
% Also check On Modeling and Rendering Ocean Scenes
% A Procedural Model for Interactive Animation of Breaking Ocean Waves
% Towards Real-Time Visual Simulation of Water Surfaces
% Realistic Water Volumes in Real-Time
% Rendering Natural Waters
% GPU-based Ocean Rendering

\subsection{Methods Used in Industry Productions}\label{subsec:methods_industry}

% FFT, Projected Grid, LOD, talk about world of warships, anno, and oceans on a
% shoestring and uncharted. Talk also about some available shaders in the Unity
% store, the default one in Unity, and in Unreal Engine, also in Cryengine.

Many different water models are used in industry productions. One that has been
applied extensively in movies and most of the games is the one from Tessendorf
\autocite{tessendorf2001simulating} as seen in \autoref{subsec:deep_water}.
During our research we came across three solutions which where presented at the
Game Developer Conference or at the SIGGRAPH conference. They are implemented in
Uncharted, World of Warships and an undisclosed game.

\subsubsection{Uncharted}\label{subsub:uncharted}
\subsubsection{World of Warships}\label{subsub:world_of_warships}
\subsubsection{Undisclosed Game}\label{subsub:undisclosed_game}

\subsection{Candidate Applications for Embedding}\label{subsec:candidate_apps}

We now present two candidate applications which need a real-time water model:
\textit{SLProject} and \textit{0 A.D.}. Although both have already one,
in the former application the model does not work and in the latter it is
outdated.


\subsubsection{SLProject}

\textit{SLProject} is a Scene Library developed and maintained by the cpvrLab at
the Bern University of Applied Sciences (BFH). It is free and open
source\footnote{Licensed under the GNU-GPL license.} and runs on Windows, MacOS,
Linux, Android and iOS\@. The project is a showcase for 3D computer graphics and
image processing topics such as real-time rendering, raytracing and feature
tracking. It is coded in C++ and OpenGL ES to ensure complete platform
independence \autocite{hudritch2017slproject, slproject2017doxygen}.

The project has a real-time water rendering demo but it is unfortunately broken
(see open ticket, \#36 ``Fix water
shader''\footnote{\url{https://github.com/cpvrlab/SLProject/issues/36}}). The
water plane is modeled by a simple multiplication of the sine and cosine
functions and is illuminated 50\% by a point light and 50\% by a cube texture.


\subsubsection{0 A.D.}

\textit{0 A.D.} is a real-time strategy game representing the 500 B.C to 500 A.D
era. It is developed by Wildfire Games, an independent game development studio.
The game is free, open source\footnote{Licensed under the GNU-GPL and CC BY-SA
license.} and runs on Windows, MacOS and Linux.

The development of the game began in 2000 when three modding groups whished to
create a free game engine. Up until 2009 the source code was accessible only to
members of Wildfire Games. Anyone interested in participating could make an
application and pass an interview. However, due to the fading interest of a
mod\footnote{Shorthand for the term \textit{modification}} which used 0 A.D.'s
game engine and the lack of programmers, they opened up the code in July 2009.
From then on many contributions have been made, notably one redesigning entirely
the code base, making new contributions easier. As of today the project is still
actively maintained, with the latest alpha-release (number 22 code named
``Venustas'') dating from July 26,
2017 \autocite{wildfire0adproject,wildfire0adstory}.

The game engine, Pyrogenesis, is heavily oriented towards modding. The core
engine is coded in C++ and Javascript is used for the game's logic, like the
artificial intelligence or the random generation of maps. This means that when
\textit{0 A.D.} is run the game engine is started with a given mod. A mod
contains all the graphical elements, 3D models, sound, shaders, scenarios, map
generators and artificial intelligence.

The documentation for developers provides good build instructions, coding
conventions and codebase descriptions. Unfortunately we found that it is
difficult for a newcomer to understand the game engine architecture. It is less
documented and scattered accross multiple pages which are sometimes outadated.
The Doxygen documentation is minimalistic and unlike the one from
\textit{SLProject}, does not help to understand the architecture.

The current water implementation has been created in 2006 by a student form the
University of Waterloo, Canada, as a course project \autocite{zaharia2006cs}.
Over the years it has been slightly modified and optimized but appears visually
outdated (see \autoref{fig:0ad_water} on page~\pageref{fig:0ad_water}) compared
to the one used in commercial RTS games like Anno 1800 or Age of Empires~3
\todo[inline]{reference figure from introduction}. There is an open ticket, \#48
``Advanced Water
Rendering''\footnote{\url{https://trac.wildfiregames.com/ticket/48}}, which was
opened twelve years ago and modified three years ago but is still not closed.
We have contacted the developers and they would gladly appreciate a contribution
that would close the issue.

\begin{figure}[t!]
    \centering
    \begin{subfigure}[t]{\textwidth}
        \centering
        \includegraphics[height=5cm]{figures/0ad_water-specular.jpg}
        \subcaption{Close view of the water.}\label{subfig:0ad_water_close}
    \end{subfigure}\\%
    %  
    \begin{subfigure}[t]{\textwidth}
        \centering
        \includegraphics[height=5cm]{figures/0ad_cycladic_arcgipelago_6.jpg}
        \subcaption{Zoomed out view.}\label{subfig:0ad_water_far}
    \end{subfigure}
    \caption{\textit{0 A.D.}'s water rendering. Notice on the lower left part
        of \autoref{subfig:0ad_water_close} the repetition of the waves.
        In \autoref{subfig:0ad_water_far} it is even more apparent (source:
        \url{play0ad.com}).}\label{fig:0ad_water}
\end{figure}
