\section{Conclusion}\label{sec:conclusion}

% In the conclusion say that it was fun, you think the result is good even if
% the application was not convincing and put a great citation about it in it.
% Say that for someone new to the graphics programming it was difficult
% Say that 16 weeks ago you knew not much about the pipeline

Although the result of our implementation did not produce convincing realistic
water, it confirmed our intuitions. The Gerstner water model has to be combined
with another one or completely discarded. Water rendering is a complex process
involving many physical properties, each having to be approximated as discussed
in \autoref{sec:water_models}. Every one of them has to be carefully selected in
order to avoid a major slowdown during rendering. Building an application from
scratch is not recommended as too many concepts like the scene graph, need to be
programmed.

As eighteen weeks ago we knew very little about the rendering pipeline and
practically nothing about its programming, we are confident to obtain good
results for our further work. We will try to implement a hybrid approach which
handles shallow and deep water into \textit{0 A.D.} using an enhanced projected
grid like the one from \autocite{kryachko2016sea}.  For the shallow water we
think to use one precomputed Gerstner wave adapted to the curvatures of the
beaches and stored as a displacement map. The fast Fourier transform from
\autocite{tessendorf2001simulating} is rather appealing to use for deep water.
This will be the main focus of our further work. If there is time left, we will
add the other elements discussed below.

We think that blending the foam in an out based on the water surface height is
amply sufficient. Interactions between objects and the water surface will have
to be placed also in a displacement texture and computed with an adapted form of
the \textit{Wave Particles} model.
